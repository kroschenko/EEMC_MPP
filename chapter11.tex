
\newpage % Эта команда начинает новую страницу
\chapter{Конкурентность}

\section{Понятие конкурентности, многопоточия и параллелизма}

Конкурентность, многопоточность и параллелизм - это понятия, связанные с одновременным выполнением нескольких задач в компьютерных системах.

Конкурентность относится к ситуации, когда несколько задач могут быть выполнены одновременно, но на самом деле происходит быстрое переключение между задачами для того, чтобы имитировать параллельную работу. Это означает, что задачи совместно используют ресурсы процессора, но могут работать независимо друг от друга. Например, веб-сервер может обрабатывать запросы от нескольких клиентов одновременно, переключаясь между ними, чтобы обеспечить отзывчивость и эффективность работы.

Многопоточность представляет собой конкурентность на уровне одного процесса, где несколько потоков могут выполняться одновременно в рамках одного процесса. Каждый поток работает независимо, но совместно использует ресурсы процессора и памяти, что позволяет ускорить выполнение задачи и улучшить отзывчивость системы.

Параллелизм - это настоящая параллельная работа, когда задачи выполняются одновременно на нескольких ядрах процессора или даже на нескольких процессорах. Это позволяет увеличить скорость выполнения задачи в несколько раз, так как каждое ядро или процессор может обрабатывать отдельную часть задачи одновременно.

Важно понимать, что конкурентность, многопоточность и параллелизм - это разные подходы к одной проблеме, и выбор подхода зависит от требований к системе и доступных ресурсов. Конкурентность может быть достаточной для системы с низкой нагрузкой, тогда как параллелизм может быть необходимым для обработки больших объемов данных или для выполнения вычислительно сложных задач.

\section{Коллекции, безопасные для конкурентного доступа}

Безопасные коллекции для конкурентного доступа - это коллекции, которые реализуют механизмы синхронизации для предотвращения одновременного доступа нескольких потоков к коллекции и обеспечения согласованного доступа к данным. Эти коллекции позволяют избежать ошибок, связанных с состоянием гонки и другими проблемами многопоточности.

В языке Java существует несколько безопасных коллекций, которые можно использовать в многопоточных приложениях. Некоторые из них:

\begin{itemize}
\item ConcurrentHashMap - это коллекция, которая позволяет безопасно добавлять, удалять и изменять элементы в хэш-таблице из нескольких потоков. Она обеспечивает механизм блокировки на уровне сегментов, что позволяет разделять таблицу на несколько частей и обеспечивает согласованный доступ к данным.
\item CopyOnWriteArrayList - это безопасная коллекция, которая обеспечивает безопасный доступ к списку из нескольких потоков. Каждый раз, когда элемент добавляется или удаляется из списка, создается новая копия списка. Таким образом, оригинальный список не изменяется, что позволяет избежать ошибок, связанных с состоянием гонки.
\item ConcurrentLinkedQueue - это безопасная коллекция, которая обеспечивает безопасный доступ к очереди из нескольких потоков. Она использует механизмы блокировки на уровне узлов, что позволяет избежать ошибок, связанных с состоянием гонки.
\end{itemize}

Эти коллекции обеспечивают безопасный и эффективный доступ к данным из нескольких потоков, и их использование может значительно упростить разработку многопоточных приложений. Однако, при выборе коллекции для конкретной задачи, необходимо учитывать требования к производительности и согласованности данных.

\section{Очереди межпоточного взаимодействия}

Очереди межпоточного взаимодействия (inter-thread communication queues) являются механизмом синхронизации, который позволяет передавать данные между потоками в многопоточном приложении. Очереди могут использоваться для обмена данными между потоками, которые выполняются параллельно или асинхронно, и могут быть реализованы в виде блокирующих или неблокирующих.

Блокирующие очереди предоставляют блокирующие операции для добавления и извлечения элементов из очереди. Когда очередь пуста, операция извлечения блокирует поток до тех пор, пока другой поток не добавит элемент в очередь. Когда очередь заполнена, операция добавления блокирует поток до тех пор, пока другой поток не извлечет элемент из очереди. Блокирующие очереди могут использоваться для реализации паттернов "Producer-Consumer" и "Worker threads".

Неблокирующие очереди, в отличие от блокирующих, не блокируют потоки, а возвращают специальные значения, указывающие на результат операции. Например, при добавлении элемента в полную очередь, операция возвращает false, а при извлечении элемента из пустой очереди, операция возвращает null. Неблокирующие очереди могут быть более эффективными, чем блокирующие, в некоторых случаях, но их использование может быть более сложным.

В Java существует несколько классов, которые реализуют очереди межпоточного взаимодействия, в том числе:

\begin{itemize}
\item BlockingQueue - это интерфейс, который определяет методы для блокирующих очередей. Некоторые реализации этого интерфейса включают ArrayBlockingQueue, LinkedBlockingQueue, SynchronousQueue.

\item ConcurrentLinkedQueue - это неблокирующая очередь, которая использует механизмы блокировки на уровне узлов.

\item TransferQueue - это интерфейс, который расширяет BlockingQueue и добавляет метод transfer(), который позволяет передавать элементы между потоками. Реализации этого интерфейса включают LinkedTransferQueue.
\end{itemize}


\section{Параллелизация вычислений}
Java предоставляет много инструментов для параллелизации вычислений, которые могут быть использованы для улучшения производительности приложения. Некоторые из них:
\begin{itemize}
\item Многопоточность - это один из наиболее распространенных способов параллелизации вычислений в Java. Многопоточность позволяет создавать несколько потоков, которые могут выполняться параллельно. В Java потоки могут быть созданы как с помощью расширения класса Thread, так и с помощью реализации интерфейса Runnable.
\item Fork/Join фреймворк - это механизм, который позволяет разбить большие задачи на более мелкие подзадачи и распределить их между потоками. Fork/Join фреймворк может быть использован для реализации рекурсивных алгоритмов, таких как сортировка слиянием или поиск с использованием древовидной структуры данных.
\item Parallel Streams - это новый механизм, введенный в Java 8, который позволяет распараллелить операции на коллекциях данных. Он использует механизмы Fork/Join фреймворка для автоматической параллельной обработки элементов коллекции.
\item Executor Framework - это механизм, который предоставляет пул потоков и механизм для отправки задач на выполнение в этом пуле. Executor Framework может быть использован для параллельной обработки независимых задач.
\item CompletableFuture API - это новый механизм, введенный в Java 8, который позволяет создавать асинхронные и параллельные вычисления с помощью Future API. CompletableFuture API предоставляет множество методов для комбинирования и синхронизации асинхронных операций.
\item Atomic Variables - это механизм, который предоставляет безопасное обновление переменных в многопоточной среде. Atomic Variables гарантируют, что операции чтения и записи будут атомарными, что позволяет избежать гонок данных.
\item ThreadLocal - это механизм, который позволяет сохранять переменные в локальном контексте потока. ThreadLocal может быть использован для сохранения состояния, которое является уникальным для каждого потока.
\end{itemize}


\begin{enumerate}% Эта команда начинает нумерованный список литературы. На каждый элемент
% списка выставляйте свою метку (с помощью команды \label{name}, где name -- имя метки),
% чтобы иметь возможность ссылаться в тексте на любой элемент списка (см. <<Руководство
% пользователю>>.

\item\label{r8} Львовский,~С.М. Набор и вёрстка в пакете~\LaTeX.~---
М., Космосинформ, 1994.

\item\label{r1} Кнут,~Д. Всё про \TeX.~--- Протвино, RD\TeX, 1993.

\end{enumerate}% Эта команда завершает нумерованный список литературы.

%----------------------------------------------------------------------------------------

\label{pages_total}% Это метка на последнюю страницу электронного учебника. Она нужна для
% корректной работы интерактивного учебника


