\newpage % Эта команда начинает новую страницу
\chapter{Тема лекции}% Эта команда начинает первую лекцию. В фигурных скобках нужно
% записать тему лекции. Эта тема лекции автоматически получит порядковый номер и
% автоматически добавится в раздел <<Содержание>>. Лекций можно создавать сколько угодно.
% Для создания новой лекции скопируйте и вставьте (или наберите с клавиатуры) команду
% \chapter{Тема лекции} в любую часть рабочего файла. Можете смело менять местами лекции --
% вся нумерация после очередной компиляции поменяется автоматически.

% При необходимости здесь можно разместить любой текст

\section{Название параграфа}% Эта команда начнет раздел, который будем далее условно называть
% параграф. В роли параграфа будет выступать структурная часть лекции (отдельный раздел,
% вопрос и т.п.). В фигурных скобках нужно вписать название параграфа. Параграф автоматически
% получит порядковый номер, который будет состоять из двух чисел, разделенных точкой.
% Первое число -- порядковый номер лекции, второе число -- порядковый номер параграфа в
% пределах лекции. Параграфов можно создавать сколько угодно. Для создания нового параграфа
% скопируйте и вставьте (или наберите с клавиатуры) команду \section{Название параграфа} в
% любую часть рабочего файла. Можете смело менять местами параграфы, переносить их из одной
% лекции  в другую -- вся нумерация после очередной компиляции поменяется автоматически.

% При необходимости здесь можно разместить любой текст

\subsection{Название подпараграфа}% Эта команда начинает подраздел, который далее будем
% условно называть подпараграфом. В фигурных скобках нужно вписать название подпараграфа.
% Подпараграф автоматически получит порядковый номер, который будет состоять из трех
% чисел, разделенных точками. Первое число -- порядковый номер лекции, второе число --
% порядковый номер параграфа, третье число -- порядковый номер подпараграфа. Подпараграфов
% можно создавать сколько угодно. Для создания нового подпараграфа скопируйте и вставьте
% (или наберите с клавиатуры) команду \subsection{Название подпараграфа} в любую часть
% Рабочего файла. Можете смело менять подпараграфы местами, переносить их из одного
% параграфа в другой (даже другой параграф другой лекции) -- вся нумерация после очередной
% компиляции поменяется автоматически.

Теперь можно приступать к формированию первой лекции Вашего учебно-методического
комплекса. Внимательно прочитайте все рекомендации <<Руководства пользователю>>.

Несколько важных моментов из <<Руководства пользователю>> мы продублируем и здесь.

Начнем с самого главного -- примера оформления программного кода в \LaTeX:

\begin{lstlisting}
    //example of simple java-program
    public class HelloWorld
    {
        public static void main(String[] args)
        {
            int i = 12;
            int a = i + 100;
            System.out.println("Hello, world!");
        }
    }
\end{lstlisting}

Определенные части текста (не содержащие формул, таблиц и иллюстраций -- о них речь пойдет
ниже) могут быть скопированы и вставлены  в рабочий документ из любого
другого текстового редактора. Исходный текст документа не должен содержать переносов
(\LaTeX~ создат их сам). Слова должны отделяться
друг от друга пробелами, но при этом \LaTeX у все-равно, сколько именно пробелов Вы
оставили между
словами, все пробелы \LaTeX~воспримет как один пробел
 (чтобы вручную управлять пробелами между словами можно использовать символ
$\sim$, который называют неразрывным пробелом).
 Конец строки также воспринимается как пробел.
Отдельные абзацы должны быть отделены друг от друга пустыми строками (опять-таки все равно,
сколько именно пустых строк стоит между абзацами, важно, чтобы была хоть одна).

Приведем пример создания определения:

\begin{opr}\label{oprperv} \rm Первообразной функции $f$ на множестве $X{\subset}D(f)$
называется такая функция $F,$ определённая на $X,$ что для любой
точки $x\in{X}$ будет выполняться равенство
  $$F'(x)=f(x).$$
\end{opr}

Пример создания теоремы:
\begin{theorem}\label{theorperv} Если функция $F$ является первообразной для функции $f$
на промежутке $X,$ то:

   {\rm a)} $F(x)+C$ также является первообразной для функции $f$ на промежутке
$X,$ где $C$ --- произвольная действительная постоянная;

    {\rm б)} для любой
другой первообразной $\Phi(x)$ функции $f$ на промежутке $X$
существует такая действительная константа $C,$ что
  $$\Phi(x)=F(x)+C.$$
\end{theorem}

Приведем пример создания гиперссылок на определение \ref{oprperv}  и теорему \ref{theorperv}.

Чтобы научиться легко создавать любые гиперссылки, внимательно прочитайте <<Руководство
пользователю>>.

Теперь приведем пример создания метки {\color{green}\hypertarget{metkatext}{на часть текста}}.

Теперь создадим гиперссылку на словосочетание <<часть текста>>. Пусть эта гиперссылка
состоит из слов <<\hyperlink{metkatext}{текстовая гиперссылка}>>

Пример создания списка:
\begin{enumerate}
\item текст;
\item текст;
\item текст.
\end{enumerate}

Пример создания следствия:
\begin{corollary}{\rm(Свойство линейности)}. Если на промежутке $I$ существуют $\int
f_k(x)dx, k=\overline{1,n}$, а $\alpha_k$ --- произвольные
действительные константы, причем хотя бы одна из них отлична от
нуля, то на $I$ существует $$\int\left(\sum_{k=1}^n \alpha_k
f_k(x)\right)dx=\sum_{k=1}^n \alpha_k
\int{f_k(x)dx}.$$\end{corollary}

Пример вставки рисунка 1n.png из папки <<pic>>/<<images>>.
\begin{figure}[h!]\center
  \includegraphics[height=5.11cm,bb=0 0 464 766]{1n.jpg}
   \caption{Тело вращения}\label{ris1}
\end{figure}

Пример вставки гиперссылки на запуск звукового файла 2.wma из папки <<media>>:
Послушаем \href{run:media/2.wma}{музыку}?

Пример вставки гиперссылки на запуск видео файла film.wmv из папки <<media>>:
\href{run:media/film.avi}{видеофильм}?

\section{Справочная информация}

Этот параграф можно назвать как угодно. Мы установим на первую страницу этого параграфа
метку \label{mybutton}, которая является <<мишенью>> Вашей кнопки на интерактивной панели
(описание процесса ее создания смотрите выше в комментариях).
\begin{enumerate}% Эта команда начинает список
\item Важная информация.
\item Очень важная информация.
\end{enumerate}% Эта команда завершает список

Если теперь скомпилировать <<Рабочий файл>>, то запустив электронный
учебник (файл UMK.pdf) и нажав на кнопку <<Ваша кнопка>> на интерактивной
панели, Вы попадете на эту страницу.

%--------------------------------------------------------------------------------------------

\section*{Вопросы и задания для самоконтроля}% Этот раздел будет состоять из вопросов
% и заданий для самоконтроля.
\addcontentsline{toc}{struct}{Вопросы и задания для самоконтроля}% Эта команда
% добавляет название раздела в раздел <<Содержание>>

Здесь можете привести список вопросов и заданий для самоконтроля:
\begin{enumerate}% Эта команда начинает список
\item Вопрос или задание.
\item Вопрос или задание.
\item Вопрос или задание.
\item Вопрос или задание.
\end{enumerate}% Эта команда завершает список

А можете подготовить с помощью программы IrenEditor (см. <<Руководство пользователю>>)
тестовое задание для самоконтроля, сохранить его в папку
<<test>>, назвав, например, lk1.exe, и создать на него метку с помощью
команды (см. <<Руководство пользователю>>): \href{run:test/lk1.exe}{Пройдем тестирование?}

%---------------------------------------------------------------------------------------------
\newpage% Эта команда задает разрыв страницы (начинает новую страницу).
\section*{ПРАКТИЧЕСКОЕ ЗАНЯТИЕ 1}% Эта команда начинает первое практическое занятие.
% Обратите внимание -- практические занятия придется нумеровать вручную.
\addcontentsline{toc}{chapter}{Практическое занятие 1 {\bf Тема занятия}} \vspace{-10pt}% Эти
% команды добавляют тему занятия в раздел <<Содержание>>
\begin{center}% Эта команда выравнивает текст по центру строки.
 {\bf% Эта команда делает шрифт полужирным
 Тема занятия}
\end{center}% Эта команда завершает <<центрирование>> текста.

{\bf Цель:} Сюда можно вписать цель практического занятия.
\\% Эта команда вставляет пустую строку.

Ваш текст.

%--------------------------------------------------------------------------------------------
\newpage% Эта команда начинает новую страницу
\section*{Задания для самостоятельного решения (выполнения)}% Эта команда начинает раздел с
% заданиями для самостоятельной работы
\addcontentsline{toc}{struct}{Задания для самостоятельного решения}% Эта команда добавляет
% название раздела в раздел <<Содержание>>

Ваш текст

\newpage% Эта команда задает новую страницу (разрыв страницы)

%--------------------------------------------------------------------------------------------
\newpage
\chapter{Организация тестирования в JUnit. Библиотека JUnit}
\section{Использование возможностей библиотеки JUnit для тестирования приложений}
JUnit - это библиотека для тестирования Java-приложений, которая предоставляет возможности для автоматизации тестирования и упрощения процесса разработки. Некоторые из возможностей, которые предоставляет JUnit для тестирования приложений, включают в себя:
    \begin{itemize}
    \item Создание и запуск тестовых кейсов: JUnit позволяет создавать тестовые кейсы, которые включают в себя один или несколько тестовых методов, которые проверяют, что код работает правильно. Тестовые кейсы запускаются автоматически при помощи JUnit Runner.

    \item Аннотации: JUnit использует аннотации для указания, какие методы являются тестами, какие методы должны быть запущены перед тестами и после них, и т.д. Например, аннотация @Test используется для указания того, что метод является тестом.

    \item Проверка результатов: JUnit предоставляет методы для проверки ожидаемых результатов тестовых методов. Например, метод \newline assertEquals() используется для проверки, что значение, возвращаемое тестовым методом, равно ожидаемому значению.

    \item Использование параметризованных тестов: JUnit позволяет создавать параметризованные тесты, которые могут использовать различные наборы входных параметров. Это упрощает тестирование кода на различных наборах данных.

    \item Использование ассертов: JUnit предоставляет более гибкие методы проверки результатов, такие как assertThat(), которые позволяют более гибко настроить проверку.

    \item Интеграция с средами разработки: JUnit может быть интегрирован с различными средами разработки, такими как Eclipse, NetBeans и IntelliJ IDEA. Это позволяет разработчикам быстро и легко запускать тесты прямо из среды разработки.

    \item Поддержка параметрических и динамических тестов: JUnit 5 добавил поддержку параметрических и динамических тестов. Это позволяет тестировать код на большом количестве различных наборов данных и генерировать тесты во время выполнения.
    \end{itemize}

\section{Организация тестов в JUnit 5}
В JUnit 5 тесты организованы в тестовые классы, каждый из которых может содержать один или несколько тестовых методов. В отличие от JUnit 4, в JUnit 5 используется аннотация @Test для указания методов, которые должны быть выполнены как тесты. Кроме того, JUnit 5 вводит новые аннотации для более гибкой настройки тестовых методов.
Пример организации тестов в JUnit 5:
\begin{lstlisting}
import org.junit.jupiter.api.Test;
import static org.junit.jupiter.api.Assertions.*;

public class CalculatorTest {

    private Calculator calculator = new Calculator();

    @Test
    public void testAddition() {
        int result = calculator.add(2, 3);
        assertEquals(5, result, "2 + 3 should equal 5");
    }

    @Test
    public void testDivision() {
        assertThrows(ArithmeticException.class, () -> {
            calculator.divide(1, 0);
        });
    }
}
\end{lstlisting}
В этом примере создается класс CalculatorTest, который содержит два тестовых метода testAddition() и testDivision(). Аннотация @Test используется для указания того, что методы являются тестами. Метод \newline testAddition() тестирует метод add() класса Calculator, а метод \newline testDivision() тестирует метод divide() и проверяет, что ожидаемое исключение \newline ArithmeticException было выброшено при попытке деления на ноль.

JUnit 5 также предоставляет новые возможности для организации тестов, такие как динамические тесты и параметрические тесты, которые могут быть использованы для генерации тестовых данных во время выполнения тестов. Кроме того, JUnit 5 поддерживает более гибкие способы настройки тестов, такие как использование аннотаций @BeforeEach, @AfterEach, @BeforeAll и @AfterAll для выполнения определенных действий перед и после выполнения тестов.

\section{Введение в методолгии Test Driven Development (TDD) и Behavior Driven Development (BDD)}
Test Driven Development (TDD) и Behavior Driven Development (BDD) являются методологиями разработки программного обеспечения, которые помогают улучшить качество и надежность кода путем активного тестирования. В Java для реализации TDD и BDD используется библиотека JUnit, которая предоставляет множество инструментов для написания и запуска тестов.

TDD в Java начинается с написания тестовых методов в JUnit, которые проверяют корректность поведения программы. Тесты должны быть написаны на языке Java и могут использовать любые API, которые доступны для этого языка. Пример тестового метода, проверяющего корректность метода сложения двух чисел:

\begin{lstlisting}
Test
public void testSum() {
    int a = 3;
    int b = 5;
    int expected = 8;
    int result = Calculator.sum(a, b);
    assertEquals(expected, result);
}
\end{lstlisting}
В этом примере мы создали тестовый метод, который вызывает метод сложения из класса Calculator и проверяет, что результат равен ожидаемому значению.

Затем разработчик должен реализовать метод сложения таким образом, чтобы тест прошел успешно. Такой подход позволяет разработчику убедиться, что его код работает корректно, и предотвратить появление ошибок в будущем.

BDD в Java использует фреймворк Cucumber для написания спецификаций. Спецификация описывает поведение системы в терминах естественного языка, который может быть понятен как для разработчиков, так и для заказчиков. Каждая спецификация представляет собой сценарий использования системы, описывающий начальные условия, действия пользователя и ожидаемые результаты.

Пример спецификации на языке Gherkin для сценария использования системы регистрации новых пользователей:

\begin{lstlisting}
Feature: User registration

Scenario: Successful registration
  Given a new user navigates to the registration page
  When they enter their details and submit the form
  Then they should be redirected to the login page
  And they should receive a confirmation email
\end{lstlisting}
В этом примере мы описали сценарий использования системы, который проверяет, что новый пользователь может успешно зарегистрироваться в системе.

Для запуска тестов BDD используется фреймворк Cucumber, который автоматически генерирует код на Java для каждой спецификации. Разработчик должен реализовать методы, которые будут вызываться при выполнении каждого шага спецификации.