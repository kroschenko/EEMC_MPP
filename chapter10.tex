\newpage

\chapter{Атомарность}

\section{Атомарность}

Атомарность в Java относится к гарантии, что операции с переменными будут выполнены как единое неделимое действие, без возможности прерывания или вмешательства других потоков.

В Java атомарность может быть обеспечена с помощью использования классов-обёрток для примитивных типов данных, таких как AtomicInteger, AtomicLong, AtomicBoolean, которые позволяют выполнять операции инкремента, декремента, сравнения и присваивания в атомарном режиме.

Кроме того, ключевое слово volatile может быть использовано для обеспечения атомарности чтения и записи переменной, гарантируя, что эти операции не будут переупорядочены в процессе оптимизации и будут видны другим потокам немедленно после выполнения.

Необходимость обеспечения атомарности возникает в многопоточных приложениях, где несколько потоков могут обращаться к одной и той же переменной одновременно, что может привести к непредсказуемому поведению, если эти операции не будут выполнены в атомарном режиме.

\section{Атомарные переменные}

Атомарные переменные в Java - это переменные, для которых операции чтения и записи выполняются атомарно, т.е. как единое неделимое действие, без возможности прерывания или вмешательства других потоков.

Java предоставляет несколько классов, которые реализуют атомарные переменные, такие как:

\begin{enumerate}
    \item AtomicInteger - атомарная целочисленная переменная.
    \item AtomicLong - атомарная переменная длинного целого типа.
    \item AtomicBoolean - атомарная логическая переменная.
    \item AtomicReference - атомарная ссылочная переменная.
    \item AtomicStampedReference - атомарная ссылочная переменная с меткой времени.
\end{enumerate}

Эти классы позволяют выполнять операции чтения и записи атомарно, в том числе инкремент, декремент, сравнение и замена значений переменных.

Использование атомарных переменных может быть полезно в многопоточных приложениях, где несколько потоков могут обращаться к одной и той же переменной одновременно. Если операции чтения и записи не выполняются атомарно, это может привести к непредсказуемому поведению приложения и нарушению согласованности данных.

\section{Когерентность памяти}

Когерентность памяти (memory coherence) - это свойство многопроцессорных или многопоточных систем, гарантирующее, что все процессоры или потоки в системе будут видеть одинаковое состояние памяти в любой момент времени.

Когерентность памяти важна для обеспечения правильной работы многопоточных программ, так как различные потоки могут обращаться к общей памяти и изменять её состояние одновременно. Если память не является когерентной, то это может привести к непредсказуемым результатам, таким как некорректное чтение данных, потеря данных, повреждение данных и т.д.

Существует несколько методов обеспечения когерентности памяти в многопроцессорных или многопоточных системах. Один из самых распространенных методов - это использование протоколов когерентности памяти (cache coherence protocols), которые гарантируют, что все процессоры или потоки видят одно и то же состояние памяти в любой момент времени. Эти протоколы используются в аппаратуре процессоров и в операционных системах для обеспечения когерентности памяти.

В Java когерентность памяти обеспечивается с помощью модели памяти Java Memory Model (JMM), которая определяет правила чтения и записи переменных для многопоточных программ. JMM гарантирует, что все потоки в системе будут видеть одно и то же значение переменной в любой момент времени, если правильно использовать ключевые слова synchronized, volatile и final.

\section{Ключевое слово volatile}

Ключевое слово volatile в Java используется для объявления переменных, которые могут быть изменены несколькими потоками и требуются для обеспечения когерентности памяти.

Когда переменная объявляется как volatile, это означает, что операции чтения и записи для этой переменной являются атомарными (т.е. выполняются как единое неделимое действие) и видны всем потокам сразу после записи. Это гарантирует, что значения переменной будут согласованными между всеми потоками, которые обращаются к этой переменной.

Ключевое слово volatile не гарантирует, что переменная будет изменяться атомарно в смысле выполнения сложных операций, таких как инкремент или декремент. Однако, это гарантирует, что операции чтения и записи для этой переменной будут выполняться в правильном порядке, т.е. чтение всегда будет происходить после записи.

Использование ключевого слова volatile может быть полезным в многопоточных приложениях, где несколько потоков могут обращаться к одной переменной одновременно, и где обеспечение когерентности памяти является критически важным. Однако, использование volatile следует ограничивать только в тех случаях, когда это действительно необходимо, так как это может повлиять на производительность приложения.

\section{Пакет java.util.concurrent.Atomics}

Пакет java.util.concurrent.atomic предоставляет классы для работы с атомарными переменными, которые могут быть изменены несколькими потоками без блокировки и обеспечивают безопасность потоков в многопоточных приложениях.

В пакете java.util.concurrent.atomic содержатся следующие классы атомарных переменных:

\begin{enumerate}
    \item AtomicBoolean - атомарная переменная типа boolean.
    \item AtomicInteger - атомарная переменная типа int.
    \item AtomicLong - атомарная переменная типа long.
    \item AtomicReference - атомарная переменная, которая ссылается на объект.
    \item AtomicStampedReference - атомарная переменная, которая ссылается на объект и имеет метку времени (timestamp).
    \item AtomicIntegerFieldUpdater - класс, который позволяет атомарно обновлять целочисленные поля в заданном классе.
\end{enumerate}

Классы из пакета java.util.concurrent.atomic обеспечивают атомарность операций чтения/записи, что означает, что каждая операция чтения или записи к атомарной переменной будет выполнена полностью, без возможности прерывания другим потоком. Это обеспечивает безопасность потоков в многопоточных приложениях.

Кроме того, классы из пакета java.util.concurrent.atomic обеспечивают оптимальную производительность в многопоточных приложениях, потому что они не блокируются при выполнении операций чтения и записи, что позволяет множеству потоков обращаться к атомарным переменным одновременно без задержки.
%---------------------------------------------------------------------------------------------
\newpage% Эта команда задает разрыв страницы (начинает новую страницу).
\section*{ПРАКТИЧЕСКОЕ ЗАНЯТИЕ 1}% Эта команда начинает первое практическое занятие.
% Обратите внимание -- практические занятия придется нумеровать вручную.
\addcontentsline{toc}{chapter}{Практическое занятие 1 {\bf Тема занятия}} \vspace{-10pt}% Эти
% команды добавляют тему занятия в раздел <<Содержание>>
\begin{center}% Эта команда выравнивает текст по центру строки.
 {\bf% Эта команда делает шрифт полужирным
 Тема занятия}
\end{center}% Эта команда завершает <<центрирование>> текста.

{\bf Цель:} Сюда можно вписать цель практического занятия.
\\% Эта команда вставляет пустую строку.

Ваш текст.

%--------------------------------------------------------------------------------------------
\newpage% Эта команда начинает новую страницу
\section*{Задания для самостоятельного решения (выполнения)}% Эта команда начинает раздел с
% заданиями для самостоятельной работы
\addcontentsline{toc}{struct}{Задания для самостоятельного решения}% Эта команда добавляет
% название раздела в раздел <<Содержание>>

Ваш текст

\newpage% Эта команда задает новую страницу (разрыв страницы)

%--------------------------------------------------------------------------------------------


\section*{Вопросы и задания для самоконтроля}% Этот раздел будет состоять из вопросов
% и заданий для самоконтроля.
\addcontentsline{toc}{struct}{Вопросы и задания для самоконтроля}% Эта команда
% добавляет название раздела в раздел <<Содержание>>

Здесь можете привести список вопросов и заданий для самоконтроля:
\begin{enumerate}% Эта команда начинает список
\item Вопрос или задание.
\item Вопрос или задание.
\item Вопрос или задание.
\item Вопрос или задание.
\end{enumerate}% Эта команда завершает список

А можете подготовить с помощью программы IrenEditor (см. <<Руководство пользователю>>)
тестовое задание для самоконтроля, сохранить его в папку
<<test>>, назвав, например, lk1.exe, и создать на него метку с помощью
команды (см. <<Руководство пользователю>>): \href{run:test/lk1.exe}{Пройдем тестирование?}

%---------------------------------------------------------------------------------------------
\newpage% Эта команда задает разрыв страницы (начинает новую страницу).
\section*{ПРАКТИЧЕСКОЕ ЗАНЯТИЕ 2}% Эта команда начинает второе практическое занятие.
% Обратите внимание -- практические занятия придется нумеровать вручную.
\addcontentsline{toc}{chapter}{Практическое занятие 2 {\bf Тема занятия}} \vspace{-10pt}% Эти
% команды добавляют тему занятия в раздел <<Содержание>>
\begin{center}% Эта команда выравнивает текст по центру строки.
 {\bf% Эта команда делает шрифт полужирным
 Тема занятия}
\end{center}% Эта команда завершает <<центрирование>> текста.

{\bf Цель:} Сюда можно вписать цель практического занятия.
\\% Эта команда вставляет пустую строку.

Ваш текст.

%--------------------------------------------------------------------------------------------
\newpage% Эта команда начинает новую страницу
\section*{Задания для самостоятельного решения (выполнения)}% Эта команда начинает раздел с
% заданиями для самостоятельной работы
\addcontentsline{toc}{struct}{Задания для самостоятельного решения}% Эта команда добавляет
% название раздела в раздел <<Содержание>>

Ваш текст

%--------------------------------------------------------------------------------------------

\newpage% Эта команда задает новую страницу (разрыв страницы)


\normalsize\section*{Варианты заданий для индивидуальной работы}% Эта команда создает новый
% раздел, название которого записано в фигурных скобках.
\addcontentsline{toc}{struct}{Варианты заданий для индивидуальной работы} % Эта
% команда добавляет название данного раздела к разделу <<Содержание>>

\begin{center}
   \bf{Вариант 1}
\end{center}

Ваш текст.

\begin{center}
   \bf{Вариант 2}
\end{center}

Ваш текст.

\begin{center}
   \bf{Вариант 3}
\end{center}

Ваш текст.

%--------------------------------------------------------------------------------------------

\newpage% Эта команда задает новую страницу (разрыв страницы)
{\large \section*{Вопросы для подготовки к экзамену и (или) зачету}% Эта команда начинает
% новый раздел, название которого записано в фигурных скобках
\addcontentsline{toc}{struct}{Вопросы для подготовки к экзамену и (или) зачету}% Эта
% команда добавляет название данного раздела к разделу <<Содержание>>

\begin{enumerate}% Эта команда начинает нумерованный список вопросов
 \item Вопрос.
\item Вопрос.
\item Вопрос.
\item Вопрос.
\end{enumerate}% Эта команда завершает нумерованный список вопросов

%--------------------------------------------------------------------------------------------

